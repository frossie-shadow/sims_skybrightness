%\documentclass[12pt,preprint]{aastex}
\documentclass{emulateapj}  %to switch to 2-column, comment out first line and uncomment 2nd line
\usepackage{url}
%\usepackage{natbib}
%\usepackage{xspace}
\def\arcsec{$^{\prime\prime}$}
\bibliographystyle{apj}
\newcommand\degree{{^\circ}}
\newcommand\surfb{$\mathrm{mag}/\square$\arcsec}
\newcommand\Gyr{\rm{~Gyr}}
\newcommand\msun{\rm{M}_\odot}
\newcommand\kms{km s$^{-1}$}
\newcommand\al{$\alpha$}
\newcommand\ha{$\rm{H}\alpha$}
\newcommand\hb{$\rm{H}\beta$}



\shorttitle{LSST Sky Model}
\shortauthors{Yoachim et al.}

\begin{document}

\title{A Sky Brightness Model for LSST}


\author{Peter Yoachim\altaffilmark{1}, author 2, author 3}

\altaffiltext{1}{Department of Astronomy, University of Washington, Box 351580,
Seattle WA, 98195; {yoachim@uw.edu} }

%\begin{abstract}
%\end{abstract}


\section{Introduction}

An accurate model of the background sky emission is needed to accurately simulate and schedule observations for LSST.  As a starting point, we use the ESO SkyCalc Sky Model Calculator\footnote{\url{http://www.eso.org/observing/etc/bin/gen/form?} \url{INS.MODE=swspectr+INS.NAME=SKYCALC}} \citep{Noll12,Jones13}.  The ESO model includes scattered moonlight, scattered starlight, zodiacal light, molecular emission from the lower atmosphere, emission lines from the upper atmosphere, and airglow continuum.  The ESO model is designed for Paranal, which seems to be very similar to Pachon.  Pachon is located 625 km south of Paranal, at an altitude of 2,700m compared to Paranal at 2,600m.  

The general idea is that the ESO model computes the expected sky background from different sources from first principles (e.g., single and double scattering of lunar light). Because LSST will want to compute the sky background for the full sky millions of times throughout the survey, we need to improve performance by pre-computing the ESO model on relevant parameter grids and interpolating the results to our specific observation times. Also, the ESO source code is not publicly available, limiting our ability to run it directly at scale.

Our new sky brightness code described in this paper can be found at \url{https://github.com/lsst/sims\_skybrightness}. In \S\ref{sec:eso} and~\ref{sec:interp}, we give a brief description of the sky components in the ESO model and how we interpolate their results. In \S\ref{sec:obs} and~\ref{sec:twi} we describe observations taken on Pachon and how we have used those observations to construct a model of the twilight sky.  Finally, \S\ref{sec:validate} shows our 

\section{The ESO Model}\label{sec:eso}
In this section, we briefly describe the sky components included in the ESO model and how we have created grids of model spectra with them. The ESO model is described in detail in \citet{Noll12} and \citet{Jones13}.

For each ESO model component, we save the output spectra as numpy zip files.  The model spectra run from 300 nm to 2 microns, with 0.1 nm stepsize.  In addition to the spectra, we use the LSST filter response curves to pre-compute the 6 LSST magnitudes for each spectrum at each grid point. These $\sim3500$ pre-computed spectra and $\sim$21,000 magnitudes are stored in their own git repository (\url{https://github.com/lsst/sims\_skybrightness\_data}) and are about 1 Gb in size.

\subsection{Zodiacal Light}
Zodiacal light is caused by sunlight scattered by dust grains in the plane of the ecliptic.  The ESO model varies the Zodiacal light as a function of ecliptic coordinates and airmass.  

We use an airmass grid of 1, 1.2, 1.4, 2.0, and 2.5 and a Healpixel \citep{Gorski99} grid with 192 elements (nside=4, resolution of $\sim15\degree$).  We use the ESO calculator to generate spectra at the center of each Healpixel coordinate and each airmass value, resulting in 960 zodiacal spectra that we can then interpolate to arbitrary position and airmass.  The Healpixels are a grid in heliocentric ecliptic latitude and longitude (i.e., longitude zero is fixed to the solar position).  Some example zodiacal light spectra are shown in Figure~\ref{fig:zodiacal}. 


\begin{figure}
  \plotone{../Plots/zodiacal.pdf}
  \caption{Example of zodiacal light interpolated from ESO templates. \label{fig:zodiacal}}
\end{figure}


\subsection{Scattered Moonlight}

\citet{Krisciunas91} provide one of the most popular models for computing the scattered moon light. This model is based on observed magnitudes from Mauna Kea. The ESO code uses the updated model of \citet{Jones13} which is fully spectroscopic and designed for Cerro Paranal. \citet{Jones13} claim their model uncertainty is $<20$\% in the optical.  Unlike the \citet{Krisciunas91} model that is a fit to observed broadband sky brightnesses, the \citet{Jones13} model uses fully 3D single scattering calculations and provides an entire spectrum. The \citet{Jones13} model does a better job matching observations than the \citet{Krisciunas91}, particularly at short wavelengths.

We build a template library of scattered moonlight by considering moon-sun separations of 0$\degree$\ to 180$\degree$ in 15 degree steps, and moon altitudes from 15$\degree$\ to 90$\degree$.  We then compute the expected spectrum on a grid of 29 positions across the sky. This is a Healpixel grid (with nside=4, resolution of $\sim15\degree$), based on altitude and azimuth (relative to the moon), where we ignore pointings with an airmass greater than 2.6. This results in a total of 2,262 template spectra for the scattered moonlight\footnote{The very observant might notice that since the azimuth is defined relative to the moon, the sky is symmetric and we should be able to drop half of the Healpixels}. Examples of the scattered moonlight spectra are shown in Figure~\ref{fig:moon}.

\begin{figure}
  \plotone{../Plots/moon.pdf}
  \caption{Example of the scattered lunar light. \label{fig:moon}}
\end{figure}


\subsection{Airglow}

The airglow is assumed to be a function of airmass and solar activity as measured by solar radio observations.  We compute the spectra on a grid of airmass of 1 to 2 with steps of 0.1 and additionally an airmass of 2.5.  For the solar activity, we span 50 to 310 SFU with a step size of 20, resulting in a total of 168 airglow spectra. 

Technically, the airglow also varies with the time of year and time of night.  This variation is quite small relative to the solar activity \citep{Noll12}, so we have not included it.  \citet{Patat08} also shows the seasonal variation is negligible in $B$, and grows to around a peak of $\pm0.5$ mags in $I$. Examples of the airglow spectra are shown in Figure~\ref{fig:airglow}.

\begin{figure}
  \plotone{../Plots/airglow.pdf}
  \caption{Example of airglow spectra. \label{fig:airglow}}
\end{figure}

  

\subsection{Scattered Star Light}

The scattered star light is a very minor component in the sky background, thus the ESO model uses a mean spectrum for the scattered starlight.  An example of the scattered star light is show in Figure~\ref{fig:merged}. This component in only a function of airmass.


\subsection{Atmospheric Emission Lines}

The airglow and emission lines vary through the night and by season, but \citet{Noll12} show the variation is at the 10-20\% level.  Since LSST is a broad band survey, such mild changes in narrow emission features should not strongly effect the integrated sky background.  For scheduling observations, the variation in background caused by changing airmass should swamp any effects from the skylines fading during the night.

The lower atmosphere only contains emission in the IR, but we have included it for completeness.  We assume the atmospheric emission is a function of only airmass and take template spectra at with airmass of 1 to 2 with steps of 0.1 and additionally an airmass of 2.5.

Because the upper atmosphere, lower atmosphere, and scattered star light only vary as a function of airmass, we have combined them. By default, we interpolate from model spectra at 12 airmass values.  Example of the spectra are shown in Figure~\ref{fig:merged}.


\begin{figure}
  \plotone{../Plots/merged0.pdf}\plotone{../Plots/merged1.pdf} \\
  \plotone{../Plots/merged2.pdf}\plotone{../Plots/merged3.pdf}
  \caption{Examples of the sky brightness components that are only dependent on airmass. \label{fig:merged}}
\end{figure}


\section{Interpolating the Templates}\label{sec:interp}

For all the interpolations, we weight and average the base-10 log of the template spectra (or average the magnitudes).  In the future, we could experiment with switching between interpolating the spectra and the log of the spectra based on how much variation there is between interpolation points.  We use simple linear interpolation (e.g., the spectrum of an observation at an airmass of 1.02 will be the sum of 0.8 times the airmass 1.0 template and 0.2 times the airmass 1.1 template). 

For the Zodiacal and Lunar components, since we have placed the templates at healpix grid-points, we can use fast healpy routines to find the 4 nearest Healpixel points, along with their weights.  


\section{Observations}\label{sec:obs}

Several sky monitoring instruments have been installed at the LSST site on Cerro Pachon.  A Canon 5D Mark III SLR has been taking data since Jan 2014.  The Canon takes 10s exposures in RGB filters throughout the night and during twilight.  More information on the Canon can be found at \url{http://ls.st/mfz}.

The current pipeline for the Canon data takes aperture photometry of bright stars and records their magnitudes as well as the sky brightness measured in an annulus around each star.  We have combined the observations into a database with a resulting 42,000 images producing 91.5 million photometric measurements spanning 311 days.  

In addition to the Canon camera, three NIST-calibrated photodiodes have been installed to observe the zenith sky brightness in $R$, $z$, and $y$ bands. The photodiode data spans 293 days, and includes 36.1 million sky brightness measurements.

We use these sky observations to fit a reasonable twilight sky brightness model and verify the other components of the sky brightness model.

\section{Additional Components}\label{sec:twi}
\subsection{Twilight}

The ESO sky model does not include a component for sunlight scattered by earth's atmosphere.  The twilight sky brightness is difficult to compute analytically.  While scattered moonlight can be computed via a single or double scattering model, the solar twilight comes from multiple scatterings, thus there is no simple analytic model for computing the solar twilight from first principles and models must instead rely on Monte Carlo radiative transfer simulations \citep{Patat06}.

Rather than compute radiative transfer, we fit a simple empirical model to the twilight flux.  Data from the all-sky camera as well as other sites show that after the sun's altitude drops below $\sim-10\degree$ the twilight flux decays exponentially with solar altitude. Figure~\ref{diodePlot} shows the zenith sky brightness as a function of sun altitude measured by the photodiodes. The total sky brightness is well fit by an exponential decay plus a constant floor.  In Figures~\ref{fig:twiSky} and~\ref{fig:twiExp} we show that when one uses alt-az coordinates, where azimuth is heliocentric (sun always at az=0), the twilight is well described as an exponential decay plus a constant for all points in the sky.

The data from the photodiodes seems to imply the twilight light is significantly fainter than previous studies have found. In Figure~\ref{fig:compareZenithDiode}, we compare the photodiode observations to the sky brightness models. At moonrise (moon altitude of 0$\deg$), we expect the sky to become significantly brighter as single-scattering of moonlight becomes possible over the visible sky.  Figure~\ref{fig:compareZenithCanon} shows that the Canon all-sky camera does see the expected sharp transition at moonrise.  It would appear that the photodiodes are not sensitive enough to detect the dark-time sky brightness level, and only register changes when the moon is very high or twilight is already well underway. Thus, the photodiode data is of limited use for fitting the free parameters in Equation~\ref{eqn:twi}.





% from fitDiode.py
\begin{figure*}
  \plotone{../../examples/Plots/diode.pdf}
  \caption{Photodiode observations of zenith at the LSST site. The light gray points show individual measurements, while the yellow points are the median-binned data. The solid blue line shows the best fit exponential decay plus constant. The green vertical line marks 12 degree twilight, and the dashed vertical blue line shows where the data was not used because the detector was often saturated at that point. \label{diodePlot}}
\end{figure*}


\begin{figure*}
  \epsscale{.3}
  \plotone{../../examples/Plots/diodeCheck_r.pdf} \plotone{../../examples/Plots/diodeCheck_z.pdf} \plotone{../../examples/Plots/diodeCheck_y.pdf} \\
  \epsscale{1}
  \caption{Comparing the photodiode zenith observations (top row) to the sky brightness model (bottom row).  The sky model predicts a sharp transition at moonrise that is not seen in the observations.   \label{fig:compareZenithDiode}}
\end{figure*}


\begin{figure*}
  \epsscale{.3}
  \plotone{../../examples/Plots/simple_zenith_comp_B.pdf} \plotone{../../examples/Plots/simple_zenith_comp_G.pdf} \plotone{../../examples/Plots/simple_zenith_comp_R.pdf}
  \epsscale{1}
  \caption{Comparing the Canon all-sky camera zenith observations (top row) to the model (bottom row).  Unlike the photodiode observations, there is an obvious transition in the observations at a moon altitude of $0\deg$. \label{fig:compareZenithCanon}}
\end{figure*}



% from twiPlotsForDoc.py
\begin{figure*}
  \plotone{../../examples/Plots/twiExamples.pdf}
  \caption{Sky brightness values measured from the Canon all-sky camera after median binning spatially by Healpixels and combining similar sun altitude observations. Only frames with the moon below the horizon were used. Zenith is at the center of each image, and the sun is below the horizon at an azimuth of zero (bottom of the frame). The values are instrumental magnitudes.  \label{fig:twiSky}}
\end{figure*}


% from fitTwiSlopesSimul.py
\begin{figure*}
  \epsscale{1}
  \plottwo{../../examples/Plots/altDecay.pdf}{../../examples/Plots/altDecayHA.pdf}
  \epsscale{1}
  \caption{Photometry from the Cannon all-sky camera, after it was been median-binned and selected for only times where the moon is down.  At low airmass (left panels), the sky brightness decays exponentially and has a small variation that is dominated by the change in airmass.  At higher airmasses (right panels), the decay is still exponential, but now is a function of both airmass and azimuth relative to the sun. \label{fig:twiExp}}
\end{figure*}



To model the broadband flux from only the scattered twilight, we first model the flux from scattered sunlight in the hemisphere pointed away from the sun as
\begin{equation}\label{eqn:twi1}
  f^{away} = f_{z} r_{12/z} 10^{a(\alpha+12^{\circ})+b(X-1)}
\end{equation}
where $\alpha$ is the altitude of the sun, $X$ is the airmass, $f_{z}$ is the flux at zenith during dark-time, and $r_{12/z}$ is the ratio of the 12-degree twilight zenith flux to the dark-time zenith flux. The full twilight flux in any direction is then given by
\begin{equation}
  \label{eqn:twi}
  f^{twi}  = \left\{
  \begin{array}{@{}ll@{}}
        f^{away}, & \text{if}\  \pi/2 < \phi < -\pi/2  \text{ or } X < 1.1\\
        f^{away} 10^{c (X-1) \cos{\phi}}, &  \text{if}\   \pi/2 > \phi >  -\pi/2 \text{ and } X > 1.1
        \end{array} \right.
\end{equation}
where $\phi$ is the azimuthal angle relative to the sun. The total sky background flux at zenith is then given by $f = f^{twi} + f_{z}$. Equation~\ref{eqn:twi} can be converted to magnitudes by:
\begin{eqnarray}
  m^{away} = m_0 -2.5a(\alpha+12^{\circ})-2.5b(X-1) \\
  m^{toward} = m^{away} -2.5c(X-1)\cos{\phi}
\end{eqnarray}
where $m_0$ is the scattered sunlight component to the 12-degree twilight surface brightness at zenith.  

By default, we only apply the twilight component for solar altitudes between -11 and -20 degrees. For solar altitude values less than -20 degrees, we find the other sky components always dominate. For altitudes greater than -11, our simple model breaks down but we expect LSST exposures would saturate well before this level.  

We have taken the sky brightness measurements from the Canon camera (when the moon is down) and median-binned the data based on sun altitude and alt-az coordinate.  The median binning effectively eliminates cloudy data.  Some example R-band twilight maps are shown in Figures~\ref{fig:twiSky} and~\ref{fig:twiExp}.  We use these data to find the best-fitting parameters for Equation~\ref{eqn:twi}.  

The best fitting parameters for $r_{12/z}$, $a$, $b$, and $c$ for each of the Canon filters and the photodiode data are listed in Table~\ref{table:canonFits}.

We use the photodiode measurements in Figure~\ref{diodePlot} to fit $r_{12/z}$, and $a$ and then assume the $b$ and $c$ terms are similar to the optical bands.  The photodiode data seems to be very faint at 12-degree twilight compared to other data sets.  The photodiode $r$ filter is much fainter than the Canon R, and the photodiode $z$ is much fainter than Johnson's $I$. 

One important lesson from the data is the counter-intuitive result that during twilight the darkest part of the sky is near zenith, not moderate airmass in the hemisphere facing away from the sun. One is better off observing at low airmass towards the sun rather than high airmass away from the sun.


%>>> import lsst.sims.skybrightness as sb
%>>> twi = sb.TwilightInterp()
%>>> twi.printFitsUsed()
\begin{deluxetable*}{c c c c c c c}
  \tabletypesize{\small }
  %\rotate
  \tablewidth{0pt}
  \tablecaption{Final parameters for equation~\ref{eqn:twi} used by the sky brightness twilight component. \label{table:canonFits}}
  
  \tablehead{\colhead{Filter} & \colhead{$r_{12/z}$} & \colhead{$a$ } & \colhead{$b$ } & \colhead{$c$ } & \colhead{$f_z,dark$} & \colhead{m$_z,dark$} \\
  & & \colhead{(1/radians)} & \colhead{(1/airmass)} & \colhead{(az term/airmass)} & \colhead{(erg/s/cm$^2$)$\times 10^8$}}
  \startdata
  B  & 8.42 & 22.96 & 0.29 & 0.30 & 3.05  &  22.35 \\
  G  & 4.14 & 22.94 & 0.30 & 0.32 & 5.50  &  21.71 \\
  R  & 2.73 & 22.20 & 0.30 & 0.33 & 8.02  &  21.30 \\
  \hline
  $z$\tablenotemark{a}  & 2.29 & 24.08 & 0.30 & 0.30 & 50.58  &  19.30 \\
  $y$\tablenotemark{a}  & 2.0 & 24.08 & 0.30 & 0.30 & 167.50  &  18.00 \\
 \hline 
 $u$\tablenotemark{b}  & 16.00 & 22.96 & 0.29 & 0.30 & 2.01  &  22.80
 \enddata
 \tablenotetext{a}{The $z$ and $y$ fits are based on $I$-band fits from the literature. The $b$ and $c$ terms are assumed to be similar to the other optical filters.}
 \tablenotetext{b}{The $u$ filter is assumed to be identical to $B$, but with a brighter $r_{12/z}$ value.}
 \end{deluxetable*}


While the parameters in Table~\ref{table:canonFits} do a reasonable job reproducing the observed magnitudes, we are also interested in generating the full spectrum of the sky.  For this, we assume the twilight is a modified solar spectrum.  We multiply a solar spectrum\footnote{\url{http://rredc.nrel.gov/solar/spectra/am0/ASTM2000.html}} by a low order polynomial such that it reproduces the expected broad band magnitudes.  One obvious shortcoming of this approach is that we have not included the effects of atmospheric transmission on the twilight sky spectrum. We could upgrade the code by using a library of twilight spectra scaled to the correct magnitudes, few spectral libraries span our full desired wavelength range.

To generate magnitudes in LSST filters we have not observed, we assume the parameters in Table~\ref{table:canonFits} are smooth functions of wavelength and interpolate the best-fit parameters to the central wavelengths of the LSST filters.

In our fits we have not used the photodiode $R$ band data, as the 12-degree twilight flux is surprisingly low ($r_{12/dark}=0.52$) and inconsistent with the Canon R fits.  This is concerning and it would be good to independently verify the $z$ and $y$ photodiode fit parameters.


\section{Validation}\label{sec:validate}

To verify the performance of the sky model, we compare it to the sky brightness observations made by the Canon all-sky camera.

% from validate.py
\begin{figure*}
  \plotone{../../examples/Plots/exampleSkys_0.pdf}
  \plotone{../../examples/Plots/exampleSkys_1.pdf}
  \plotone{../../examples/Plots/exampleSkys_2.pdf}
  \plotone{../../examples/Plots/exampleSkys_3.pdf}
  \caption{Some examples of Canon all-sky observations and model values for airmasses less than 2.1. The sky observations have been binned into alt-az Healpixels (zenith at the center of the projections). These are all for the Canon R-filter. The top row shows a clear dark-time frame, the second row is a dark time frame where there were clouds. The third row shows a high moon, and the final row is during twilight with some light clouds. \label{fig:skyExamples}}
\end{figure*}



Figure~\ref{fig:darkDir} shows how well the sky model does at predicting the direction of the darkest region of the sky.  For the observations from the all-sky camera, we select those that have at least 200 stars detected to select those without extreme cloud coverage. To speed things up, we use every 10th frame from the sky camera, resulting in around 3400 unique observations.  We then bin the sky brightness observations onto a Healpixel grid of nside=16 and compute the model sky brightness at each Healpixel and find the distance between the predicted brightness minimum and the observed minimum.  The results, binned by moon and sun altitude are shown in Figure~\ref{fig:darkDir}.  The results are promising, with the median offsets tending to be around 15$\degree$.  Note we have not done any significant screening for sparse clouds in the observations, which would probably reduce the differences even more. 

Using the same set of 3400 observations, we zeropoint each frame (to correct for any average cloud cover or instrument zeropoint drift) and record the model and observation zenith sky brightness. Figure~\ref{fig:zenithModel} shows the median and standard deviation of the model and observation zenith residuals.  In general, the model does an excellent job matching the observed zenith sky brightness, and only fails when the moon is very high in the sky.  




% from validate.py
\begin{figure*}
  \epsscale{.8}
  \plottwo{../../examples/Plots/medianAngDiff_R_.pdf}{../../examples/Plots/rmsAngDiff_R_.pdf}
  \plottwo{../../examples/Plots/medianAngDiff_G_.pdf}{../../examples/Plots/rmsAngDiff_G_.pdf}
  \plottwo{../../examples/Plots/medianAngDiff_B_.pdf}{../../examples/Plots/rmsAngDiff_B_.pdf}
  \epsscale{1}
  \caption{Comparing the angular distance between the model-predicted darkest spot on the sky and the darkest spot as observed by the Canon all-sky camera \label{fig:darkDir}.  Dashed lines mark moonrise and end of twilight (lower left region is dark-time).  The panels on the left show the median distance between the model and observation for each bin while the right panels show the robust-RMS for each bin.  Most of the variation can be attributed to frames with clouds.}
\end{figure*}

\clearpage


% from validate.py
\begin{figure*}
  \epsscale{0.8}
  \plottwo{../../examples/Plots/zenithMedian_R_.pdf}{../../examples/Plots/zenithRMS_R_.pdf}
  \plottwo{../../examples/Plots/zenithMedian_G_.pdf}{../../examples/Plots/zenithRMS_G_.pdf}
  \plottwo{../../examples/Plots/zenithMedian_B_.pdf}{../../examples/Plots/zenithRMS_B_.pdf}
  \epsscale{1}
  \caption{ The difference between the model predicted zenith sky brightness and the observed values from the Canon all-sky camera.  The model does tend to fail very near the moon, thus the large residuals when the moon reaches high is expected.  \label{fig:zenithModel}}
\end{figure*}


Our results can be compared to the ESO-Paranal twilight sky observations presented in \citet{Patat06}. We have scrapped the twilight zenith sky brightness data for observations taken of long-exposure standards with the FORS1 instrument.  The data and best fitting exponential-plus-constant models are shown in Figure~\ref{fig:Patat}.  The best-fit parameters are listed in Table~\ref{table:PatatFits}.  The Canon-B brightness seems to be fainter than Johnson's $B$.  The slopes of the twilight decay are very similar with the \citet{Patat06} data having $ -1.06  < a < -0.97 $ mag degree$^-1$ and our fits are in the range $ -1.02 < a < -0.97$ mag degree$^-1$.



XXX--add plots showing the dark-time residuals as a function of time-of-year and time-of-night for each filter.  See if there are any obvious residuals.--this might be getting wiped out by the way I zero-point the Canon maps?

XXX-Need a statement on how good the model is (0.1 mag, 0.5 mag etc).  Maybe just make a table out of figure 12. say, dark time, moon up, twilight. Note that clouds have not been explicitly excluded.


In Table~\ref{table:darkSky}, we compare the median dark sky surface brightness computed from our model to the values addopted in \citet{Ivezic08}.  The only variation in included in the model is from the changing impact of Zodiacal light.  Other than the $y$-band which disagrees by 0.5 magnitures per square arsecond, the model and \citet{Ivezic08} agree well.

% from checkDarkSky.py
\begin{deluxetable}{c c  c}
  \tabletypesize{\small }
  %\rotate
  \tablewidth{0pt}
  \tablecaption{Dark sky surface brightnesses \label{table:darkSky}}
  
  \tablehead{\colhead{Filter} & \colhead{Model} & \colhead{\citet{Ivezic08}} \\
  & \colhead{($\mu$)} &  \colhead{($\mu$)}  }
  \startdata
  $u$ &    22.81 $\pm$  0.04  &  22.90 \\
  $g$ &    22.27 $\pm$  0.11  &  22.30 \\
  $r$ &    21.25 $\pm$  0.08  &  21.20 \\
  $i$ &    20.38 $\pm$  0.04  &  20.50 \\
  $z$ &    19.40 $\pm$  0.02  &  19.60 \\
  $y$ &    18.10 $\pm$  0.01  &  18.60 
\end{deluxetable}



\begin{figure}
  \plotone{../PatatTwi/patatFits.pdf}
  \caption{Zenith sky surface brightness data scraped from \citet{Patat06} and fit with our simple exponential decay plus constant model. Dashed black lines show the best-fit curves.  The red dashed line shows the 12-degree twilight line.  \label{fig:Patat} }
\end{figure}


\begin{deluxetable}{c c c}
  \tabletypesize{\small }
  %\rotate
  \tablewidth{0pt}
  \tablecaption{Parameter fits to data from \citet{Patat06} \label{table:PatatFits}}
  
  \tablehead{\colhead{Filter} & \colhead{$r_{12/z}$} & \colhead{$a$ } \\
  & & \colhead{(1/radians)}  }
  \startdata
  U & 13.40 & 22.37 \\
  B & 15.49 & 22.60 \\
  V & 4.78  & 22.31 \\
  R & 3.35  & 24.35 \\
  I & 2.29  & 24.08
\end{deluxetable}


\section{Possible Future Work}

Note that we have not verified the off-zenith IR performance of the model. If the source of scattering of IR sunlight in the atmosphere is significantly different from optical light, the model could be wrong. Measuring the IR surface brightness of the twilight sky along just a few more alt-az pointings could help in verifying and constraining the $b$ and $c$ terms of Equation~\ref{eqn:twi}.

We have not included the sky brightness increase that can come from light reflected off of clouds.  With the data from the all-sky camera, we could construct a model for the increased brightness we expect as a function of atmospheric transparency.  

In theory, we should be able to use the photodiode data to verify the sky brightness model has the correct zeropoint. In practice, it has proven difficult to find anyone who knows how to convert the output of the photodiodes to magnitudes per square arcsecond.

When fitting the all-sky twilight data, we assumed that as long as the moon had set we could approximate the sky background at a given airmass as constant. We should go back and remove the expected zodiacal light as well as airglow given the solar activity at the time of the observations. Thus, we should expect the the fits in Table~\ref{table:canonFits} to be slightly skewed by the presence of zodiacal light, particularly parameter $c$.  

When a spectrum is requested from the twilight model, we currently return a solar spectrum multiplied by a smooth spline curve to match the expected broad-band magnitudes.  While the twilight is solar-like, our simple procedure means the spectrum does not have the correct atmospheric absorption features.  A future improvement could be to use observed twilight spectra to make our returned results more realistic.

The sky model currently uses the pyEphem package to compute sun and moon positions as well as a few coordinate transformations. It would be preferable to eliminate this dependency.  


\bibliography{skyModel.bib}
\end{document}



%%  LocalWords:  ESO SkyCalc airglow Noll numpy nm airmass Healpixel
%%  LocalWords:  nside Healpixels Krisciunas Patat twi az skyModel Gb
%%  LocalWords:  SIM Cerro Pachon photodiodes airmasses radians pre
%%  LocalWords:  photodiode zeropoint skybrightness stepsize Mauna
%%  LocalWords:  Kea Paranal brightnesses pointings SFU annulus NIST
%%  LocalWords:  radiative FORS healpix healpy pyEphem BVRI Gorski
